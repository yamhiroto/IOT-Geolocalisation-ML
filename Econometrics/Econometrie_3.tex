% documentclass
% set font size=11 (11pt)
% set paper format=A4 (a4paper)
% set equation alignment to left (fleqn)
\documentclass[11pt,a4paper,fleqn]{article}


% Preamble
% use the inputenc and fontenc packages to use French accents
\usepackage[utf8]{inputenc}
\usepackage[T1]{fontenc}
% for matrices / vectors
\usepackage{amsmath}
% allow for arbitrary font size
\usepackage{anyfontsize}
% for color
\usepackage{xcolor}
% for pseudocolor
\usepackage{algorithm,algpseudocode}
% for code samples
\usepackage{listings}
% set the font as Time New Roman (the Latex equivalent, at least)
% \usepackage{mathptmx}
% set the size of the document margins using the geometry package
\usepackage[lmargin=0.97in,rmargin=0.97in,tmargin=1.4in,bmargin=1.4in]{geometry}
% turn the color of footnote markers to black
\renewcommand\thefootnote{\textcolor{black}{\arabic{footnote}}}
% suppress indents on footnotes
\usepackage[hang,flushmargin]{footmisc}
% automatically generates colored brackets around references
\usepackage{fncylab} \labelformat{equation}{(#1)}
% supress indent on new paragraphs
\setlength{\parindent}{0pt}
% use the amsmath package to include mathematical symbols
\usepackage{amsmath}
% suppress the space between the left margin and the equations (fleqn still leaves some space by default)
\setlength{\mathindent}{0pt}
% create a new environment to left flush the equation with the align environment
\makeatletter
\newenvironment{lflalign}{ \vspace{-3mm}%
  \def\align@preamble{%
    &\strut@
    \setboxz@h{\@lign$\m@th\displaystyle{####}$}%
    \ifmeasuring@\savefieldlength@\fi
    \set@field
    \hfil
    \tabskip\z@skip
    &\setboxz@h{\@lign$\m@th\displaystyle{{}####}$}%
    \ifmeasuring@\savefieldlength@\fi
    \set@field
    \hfil
    \tabskip\alignsep@
  }
  \flalign}
{\endflalign}
\makeatother
% use the ammssymb package to use mathematical symbols
\usepackage{amssymb}
% create new commands for mathematical symbols
\DeclareMathOperator{\N}{\mathbb{N}}
\DeclareMathOperator{\Z}{\mathbb{Z}}
\DeclareMathOperator{\Q}{\mathbb{Q}}
\DeclareMathOperator{\R}{\mathbb{R}}
\DeclareMathOperator{\Pb}{\mathbb{P}}
% declare the cmsy (computer modern symbol) math alphabet to define appropriate fonts for the U and N mathematical symbols
\DeclareMathAlphabet\mathbcal{OMS}{cmsy}{m}{n}
% create new commands for mathematical symbols
\DeclareMathOperator{\E}{\mathbcal{E}}
\DeclareMathOperator{\Ex}{\mathbb{E}}
\DeclareMathOperator{\F}{\mathbcal{F}}
\DeclareMathOperator{\G}{\mathbcal{G}}
\DeclareMathOperator{\M}{\mathbcal{M}}
\DeclareMathOperator{\HH}{\mathbcal{H}}
\DeclareMathOperator{\QQ}{\mathbcal{Q}}
\DeclareMathOperator{\PP}{\mathbcal{P}}
\DeclareMathOperator{\Noo}{\mathbcal{N}}
\DeclareMathOperator{\U}{\mathbcal{U}}
% use the bbm package to be able to use the double stroke 1 for the indicator function
\usepackage{bbm}
\DeclareMathOperator{\ind}{\mathbbmss{1}}
% use the bm package to use bold characters in math mode
\usepackage{bm}
% create a new command for black square bullets
\newcommand{\bs}{\scalebox{0.7}{$\blacksquare$} \hspace{2mm}}
% use the relsize package to be abe to change the size of mathematical symbols
\usepackage{relsize}
% define a new command for in-line small summation
\newcommand{\ssumm}[2]{\underset{\scriptscriptstyle #1}{\overset{\scriptscriptstyle #2}{\mathlarger{\mathlarger{\mathlarger{\Sigma}}}}} \hspace{0.5mm}}
% define a new command for in-line small products
\newcommand{\sprod}[2]{\underset{\scriptscriptstyle #1}{\overset{\scriptscriptstyle #2}{\mathlarger{\mathlarger{\mathlarger{\Pi}}}}} \hspace{0.5mm}}
% Use the caption package to customize captions (titles) of tables and graphs
\usepackage[font=small,labelfont=bf]{caption}
% use float package to force figure the be positioned where indicated
\usepackage{float}
% use the graphicx package to be able to resize tables
\usepackage{graphicx}


\begin{document}

% command to check unused bibliography entries
% \nocite{*}

{\fontsize{12pt}{22pt} \textbf{Tests d'hypothèse}\par}
\vspace{5mm}
2 variables explicatives $\beta_0$, $\beta_1$

Test: $\mathcal{H}_0=0$

Si $\mathcal{H}_0$ est vraie, alors l'estimateur est sans biais $\mathbb{E}[\beta_1]=\beta_1$

Un problème commun c'est quand plusieurs variables explicatives sont corrélées (ex: les talents d'un joueur de baseball). Le test de Student n'est pas efficace.

Pour parer à ce problème on fait un test de Fisher. 

$SCE = \Sigma_iu_i^2=u'u$

$SCE$ est la variable d'ajustement.

On impose 2 contraintes: $\beta_1=0$, $\beta_2=0$ puis on calcule la statistique de Fisher:

$$F=\frac{(SCE_c-SCE_{NC})/2}{SCE_{NC}/(n-3)}$$

Numérateur: diviser par le nombre de contraintes

(Rapport de 2 Chi-2 divisées par leur degrés de liberté respectifs).

\vspace{5mm}

Si hypothèse vraie, la contrainte ne change rien => stat de Fisher proche de zéro. Pour savoir si la stat est assez proche de zéro, on fixe un seuil: si on est à droite du quantile à 95\%, on rejette l'hypothèse que les 3 paramètres sont égaux à zéro. Par Fisher, on en conclut que \textbf{les variables sont donc significatives par Fisher}. On voit qu'on ne peut pas regarder uniquement les variables individuelles pour savoir si on garde les variables.

\vspace{5mm}

On remarque que $F \sim T^2_{n-k}$

\vspace{5mm}

\underline{Cas variables qualitatives ou discrètes}

Problème: les variables ne sont pas continues. Ex: si on travaille sur des salaires, le niveau minimum est minoré. Problème de \textbf{censure}: le modèle va considérer une gaussienne globale. On aura un biais d'estimation.

\vspace{5mm}

En mettant une variable binaire pour le genre, c'est équivalent à faire l'hypothèse que la pente du salaire des hommes et femmes est constant. La différence de salaire se fait uniquement sur l'ordonnée à l'origine.

Lorsque la somme de $k$ variables est égale à $1$, on supprime une variable pour éviter la colinéarité.

\vspace{5mm}

Autre exemple:

$MBR = \beta_0 + \delta_1 CR_1 + \delta_2 CR_2 + \delta_3 CR_3 +\delta_4 CR_4 + autres facteurs$

$\delta_1$ est la différence entre avoir un $CR_1$ et $CR_5$ ($CR_5$ omis pour ne pas avoir la colinéarité (voir exemple précédent).

\vspace{5mm}

Autre exemple:

$log(wage) = \beta_0 + \delta_0 female + \beta_1 educ + \delta_1 female*educ + u$

Ici on a rajouté la variable d'intéraction pour ne plus avoir une croissance constante. En revanche le test de significativité devient plus compliqué, il faut désormais faire un test de Fisher.

Le test de Fisher est rejeté, on conclut donc que l'hypothèse d'une variation constante des salaires (homme ou femme) était réaliste.

\end{document} 
